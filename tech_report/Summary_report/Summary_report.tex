\documentclass[9pt,a4paper,]{extarticle}

\usepackage{f1000_styles}

\usepackage[pdfborder={0 0 0}]{hyperref}

\usepackage[numbers]{natbib}
\bibliographystyle{unsrtnat}


%% maxwidth is the original width if it is less than linewidth
%% otherwise use linewidth (to make sure the graphics do not exceed the margin)
\makeatletter
\def\maxwidth{ %
  \ifdim\Gin@nat@width>\linewidth
    \linewidth
  \else
    \Gin@nat@width
  \fi
}
\makeatother


% disable code chunks background
%\renewenvironment{Shaded}{}{}

% disable section numbers
\setcounter{secnumdepth}{0}

\setlength{\parindent}{0pt}
\setlength{\parskip}{6pt plus 2pt minus 1pt}



\begin{document}
\pagestyle{front}

\title{Technical report: RNA-seq differential expression analysis on EBV infected cell lines}

\author[1]{Rene Welch}
\author[2]{Sunduz Keles}
\affil[1]{Department of Statistics, University of Wisconsin-Madison}
\affil[2]{Department of Statistics, and Department of Biostatistics and Medical Informatics, University of Wisconsin-Madison}

\maketitle
\thispagestyle{front}

\begin{abstract}
Abstracts should be up to 300 words and provide a succinct summary of the article. Although the abstract should explain why the article might be interesting, care should be taken not to inappropriately over-emphasise the importance of the work described in the article. Citations should not be used in the abstract, and the use of abbreviations should be minimized.
\end{abstract}

\section*{Keywords}
EBV, RNA-seq


\clearpage
\pagestyle{main}

\hypertarget{introduction}{%
\section{Introduction}\label{introduction}}

The introduction provides context as to why the software tool was developed and what need it addresses. It is good scholarly practice to mention previously developed tools that address similar needs, and why the current tool is needed.

\hypertarget{methods}{%
\section{Methods}\label{methods}}

\hypertarget{quality-control}{%
\subsection{Quality control}\label{quality-control}}

\hypertarget{alignment-and-expression-quantification}{%
\subsection{Alignment and expression quantification}\label{alignment-and-expression-quantification}}

\hypertarget{differential-expression-analysis}{%
\subsection{Differential expression analysis}\label{differential-expression-analysis}}

\hypertarget{section}{%
\subsection{}\label{section}}

\hypertarget{results}{%
\section{Results }\label{results}}

This section is only required if the paper includes novel data or analyses, and should be written as a traditional results section.

\hypertarget{summary}{%
\section{Summary }\label{summary}}

This section is required if the paper does not include novel data or analyses. It allows authors to briefly summarize the key points from the article.

\hypertarget{data-availability}{%
\section{Data availability }\label{data-availability}}

Please add details of where any datasets that are mentioned in the paper, and that have not have not previously been formally published, can be found. If previously published datasets are mentioned, these should be cited in the references, as per usual scholarly conventions.

\hypertarget{software-availability}{%
\section{Software availability}\label{software-availability}}

This section will be generated by the Editorial Office before publication. Authors are asked to provide some initial information to assist the Editorial Office, as detailed below.

\begin{enumerate}
\def\labelenumi{\arabic{enumi}.}
\item
  URL link to where the software can be downloaded from or used by a non-coder (AUTHOR TO PROVIDE; optional)
\item
  URL link to the author's version control system repository containing the source code (AUTHOR TO PROVIDE; required)
\item
  Link to source code as at time of publication (\emph{F1000Research} TO GENERATE)
\item
  Link to archived source code as at time of publication (\emph{F1000Research} TO GENERATE)
\item
  Software license (AUTHOR TO PROVIDE; required)
\end{enumerate}

\hypertarget{author-information}{%
\section{Author information}\label{author-information}}

In order to give appropriate credit to each author of an article, the individual contributions of each author to the manuscript should be detailed in this section. We recommend using author initials and then stating briefly how they contributed.

\hypertarget{competing-interests}{%
\section{Competing interests}\label{competing-interests}}

All financial, personal, or professional competing interests for any of the authors that could be construed to unduly influence the content of the article must be disclosed and will be displayed alongside the article. If there are no relevant competing interests to declare, please add the following: `No competing interests were disclosed'.

\hypertarget{grant-information}{%
\section{Grant information}\label{grant-information}}

Please state who funded the work discussed in this article, whether it is your employer, a grant funder etc. Please do not list funding that you have that is not relevant to this specific piece of research. For each funder, please state the funder's name, the grant number where applicable, and the individual to whom the grant was assigned. If your work was not funded by any grants, please include the line: `The author(s) declared that no grants were involved in supporting this work.'

\hypertarget{acknowledgments}{%
\section{Acknowledgments}\label{acknowledgments}}

This section should acknowledge anyone who contributed to the research or the article but who does not qualify as an author based on the criteria provided earlier (e.g.~someone or an organization that provided writing assistance). Please state how they contributed; authors should obtain permission to acknowledge from all those mentioned in the Acknowledgments section.

{\small\bibliography{sample.bib}}

\end{document}
